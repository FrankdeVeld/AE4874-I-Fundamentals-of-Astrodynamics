\section{Introduction}



\subsection{Importing/synchronizing variables with latex}
\begin{enumerate}
    \item include package: usepackage{listings} %Allows import of lists (withs specific lines)
    \item create .txt file with every value you want somewhere (in an equation) on a separate line
    \item Add description of variable in the line above that variable, and (line)-number that description
    \item create a folder in your latex working environment, e.g. named: $"data\_input"$
    \item upload e.g. "test.txt" with the variables and descriptions
    \item To input the parameter in latex use the following line:
    %\lstinputlisting[firstline=2,lastline=2]{data_input/test.txt}
    \item Party!
\end{enumerate}

\subsection{Importing/synchronizing variables with latex V1.0}
    \begin{enumerate}
    \item add the package listed in the comment, then add the lines below the comment
%\usepackage{datatool,siunitx}
    \item Then upload the .txt file that contains the variables.

\DTLloaddb[noheader]{data}{testfile.txt}
\newcommand{\mydatavalue}[1]{%
  \begingroup
  \dtlgetrow{data}{#1}%
  \dtlgetentryfromcurrentrow{\temp}{1}%
  \num{\temp}%
  \endgroup
}
    \item and call it with 
    \item $K_{sh} = \mydatavalue{2}$
    \item $K_{sh} = \mydatavalue{4}$
\end{enumerate}