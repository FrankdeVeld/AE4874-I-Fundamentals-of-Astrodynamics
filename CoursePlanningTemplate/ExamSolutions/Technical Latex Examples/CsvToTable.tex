%Source0:https://www.latex-tutorial.com/tutorials/pgfplotstable/
%Source1:https://tex.stackexchange.com/questions/35929/using-string-variable-with-latex
% \usepackage{booktabs} % For \toprule, \midrule and \bottomrule
% \usepackage{siunitx} % Formats the units and values
% \usepackage{pgfplotstable} % Generates table from .csv

% % Setup siunitx:
% \sisetup{
%   round-mode          = places, % Rounds numbers
%   round-precision     = 2, % to 2 places
% }

% \begin{table}[h!]
% \begin{adjustwidth}{-2cm}{}
%   \begin{center}
%     \caption{Autogenerated table from .csv file.}
%     \label{table1}
%     \pgfplotstabletypeset[
%       multicolumn names, % allows to have multicolumn names
%       col sep=comma, % the seperator in our .csv file
%       display columns/0/.style={
% 		column name=$Value 1$, % name of first column
% 		column type={S},string type},  % use siunitx for formatting
%       display columns/1/.style={
% 		column name=$Value 2$,
% 		column type={S},string type},
%       every head row/.style={
% 		before row={\toprule}, % have a rule at top
% 		after row={
% 			\si{\ampere} & \si{\volt}\\ % the units seperated by &
% 			\midrule} % rule under units
% 			},
% 		every last row/.style={after row=\bottomrule}, % rule at bottom
%     ]{csvTables/table.csv} % filename/path to file
%   \end{center}
% \end{table}

\section{retry}
% Put this in Main.tex above \begin document
% \usepackage{csvsimple}
% \usepackage{geometry}
%  \geometry{
%  a4paper,
%  total={175mm,265mm},
%  left=15mm,
%  top=15mm,
%  }
 
 
\section{hi}
\hspace*{-2em}
\begin{tabular}{l|l|l|l|l|l|l|l|l}%
    \bfseries Nr. & \bfseries Cal & \bfseries tw & \bfseries Topic & \bfseries Available & \bfseries Due & \bfseries Source due & \bfseries Weight & \bfseries Source weight % specify table head
    % \csvreader[head to column names]{Basket_ball.csv}{}% use head of csv as column names
    \csvreader[head to column names]{CsvTables/table.csv}{}% use head of csv as column names
    {\\\hline\csvcoli&\csvcolii&\csvcoliii&\csvcoliv&\csvcolv&\csvcolvi&\csvcolvii&\csvcolviii&\csvcolix}% specify your coloumns here
    \end{tabular}
    
