\section{Exercise 3}
We look at the movement of a satellite around the Earth. From observations of the movements of the satellite, we have derived that the orbit of the satellite has a period of $T = 1.85$ hours, or $T = 6660 s$ en hat on a given time $T_0$ the radial and normal components of the velocity of the satellite are given by $\dot{r} = -2481 m/s$ and $r \dot{Theta} = 6572.6 m/s$. Furthermore, it is known that the satellite has a low eccentricity; $e<0.3$.

\subsection*{a.}
It is asked to derive expressions for the elliptical orbit from the general equations.
The general equations are

\begin{equation}
    \begin{split}
        H = r^2 \dot{\theta} \\
        r = \frac{H^2/\mu}{1+e \cos{(\theta)}} \\
    \end{split}
\end{equation}

From this we can see:
\begin{equation}
    r \dot{\theta} = \frac{H}{r} = \frac{H}{\frac{H^2/\mu}{1+e \cos{(\theta)}}}
\end{equation}

Simplification gives:

\begin{equation}
    r \dot{\theta} = \frac{\mu}{H}(1+e\cos{\theta})
\end{equation}

Taking the derivative of $r$ gives:

\begin{equation}
    \dot{r} = \frac{H^2}{\mu}(e\sin{(\theta)})\frac{\dot{\theta}}{(1+e\cos{(\theta)})}
\end{equation}

Now since $\dot{\theta} = \frac{H}{r^2}$, we have:

\begin{equation}
    \dot{r} = \frac{H^2}{\mu}(e\sin{(\theta)})\frac{1}{(1+e\cos{(\theta)})} \frac{1}{\frac{(H^4/\mu^2)}{(1+e\cos{(\theta)})^2}}
\end{equation}

This simplifies to:

\begin{equation}
    \dot{r} = \frac{\mu}{H}e \sin{(\theta)}
\end{equation}

\subsection*{b.}
From Kepler's Third Law, we have 
\begin{equation}
    a = (\frac{T^2 \mu}{4 \pi^2})^{1/3}
\end{equation}

Filling in the values gives $a = 7.651 \cdot 10^6$ m. We also know that $H = \sqrt{p \mu} = \sqrt{a(1-e^2) \mu}$. Together, these equations can be solved iteratively (or with a MatLab equation solver), but I did not manage to obtain values.
% I tried Solve[{398600.4*10^9/(sqrt(398600.4*10^9*(7.651*10^6*(1-e^2))))*e*sin(theta) == -2481, 398600.4*10^9/(sqrt(398600.4*10^9*(7.651*10^6*(1-e^2))))*(1+e*cos(theta))==6572.6},{e, theta}] in Wolfram Alpha, but it did not work, reason unclear.


\subsection*{c.}
With the eccentricity obtained, we can not that the perihelion is given by $r_p = a - ae = a(1-e)$. To compute the height above surface level, the formula $h_p = a(1-e) - R_e$ is used.
Similarly, for the aphelion we have $r_a = a + ae = a(1+e)$ and $h_a = a(1+e) + R_p$.


\subsection*{d.}
To get the parameter $t_0 - \tau$, we first introduce the eccentric anomaly $E$ in the following formula:

\begin{equation}
    r \cos{(\theta)} = a\cos{(E)} - ae
\end{equation}

Alternatively, we have the expression

\begin{equation}
    r \sin{(\theta)} = a \sqrt{1-e^2} \sin{(E)}
\end{equation}

Summing these formulae squared then gives:

\begin{equation}
    r = a (1-e\cos{(E)})
\end{equation}

This is positive since the negative value has no meaning.
Differentiation to time gives:

\begin{equation}
    \dot{r} = a e \dot{E} \sin{(E)}
\end{equation}

We also have still the relation

\begin{equation}
    r = \frac{a(1-e^2)}{1+e\cos{(\theta)}}
\end{equation}

From this, we can see that:

\begin{equation}
    \dot{r} = \frac{\mu e \sin{(\theta)}}{\sqrt{\mu a (1-e^2)}}
\end{equation}

And thus:

\begin{equation}
    \frac{\mu e \sin{(\theta)}}{\sqrt{\mu a(1-e^2)}} = a e \dot{E} \sin{(E)}
\end{equation}

Substituting $r \sin{(\theta)} = a \sqrt{1-e^2}\sin{(E)}$ gives:

\begin{equation}
    \frac{\mu e a \sqrt{1-e^2}}{\sqrt{\mu a(1-e^2)} r} = a e \dot{E}
\end{equation}

Since $r = a(e-e\cos{(E)})$, this leads to:

\begin{equation}
    \frac{\mu \sqrt{1-e^2}}{\sqrt{\mu a(1-e^2)}(1-e\cos{(E)})} = a \dot{E}
\end{equation}

Which gives:

\begin{equation}
    \frac{\mu}{\sqrt{\mu a}} = a \dot{E} (1-e\cos{(E)})
\end{equation}

Integration gives:

\begin{equation}
    t-\tau = \sqrt{a^3}{\mu} (E-e\sin{E})
\end{equation}

Also, we have the transformation relation:

\begin{equation}
    \tan{(\theta/2)} = \sqrt{\frac{1+e}{1-e}}\tan{(\frac{E}{2})}
\end{equation}

From this, the value of E can be obtained (if indeed you get values for e and $\theta$).

% From $H = r^2 \dot{\theta}$ we have:
% \begin{equation}
%     \frac{d \theta}{d t} = \frac{H}{r^2} = \frac{\sqrt{\mu}p}{r^2}
% \end{equation}

% Then we get

% \begin{equation}
%     dt = \frac{p^2}{\sqrt{\mu p}} \frac{d \theta}{(1+e \cos{(\theta)})^2}
% \end{equation}

% This then gives:

% \begin{equation}
%     t - \tau = \sqrt{\frac{p^3}{\mu}} \int_{0}^{\theta} \frac{d\theta'}{(1+\cos{(\theta')})^2}
% \end{equation}

