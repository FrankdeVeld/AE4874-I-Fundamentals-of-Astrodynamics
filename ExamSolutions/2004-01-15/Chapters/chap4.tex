\section{Exercise 4}
\subsection*{a.}
The following terms need to be explained:
\begin{itemize}
    \item Equatorial plane: Plane through the equator (of the Earth), perpendicular to the rotation axis and equidistant to both poles, extended into infinity.
    \item Meridian: Half of an imaginary great circle parallel to the Greenwich Meridian, with start and end points on the North and South Pole.
    \item Geographical Distance: distance on a sphere, so an arc distance. It is a distance measured on the surface of the Earth. Geographical length: length of an arc parallel to the equator on a sphere's surface.
    \item Geocentric Width: Length of an arc along a meridian on the celestial sphere, with the Earth as a center point.
    \item Standard Ellipsoid: Mathematical figure representing the approximation for the shape of a celestial body.
    \item Geodetic width: shortest distance between two points on a sphere with the same distance to the equator.
\end{itemize}

\subsection*{b.}
The following terms need to be explained:
\begin{itemize}
    \item Celestial sphere: abstract sphere with an arbitrary large radius and its origin at the Earth's center (or another celestial body).
    \item Hour circle: The great circle through both celestial poles and the body of interest.
    \item Declination: Angular distance between a body and its celestial equator.
    \item Right Ascension: Angular distance between a body and the vernal equinox.
    \item Ecliptic: Plane on which a celestial body moves on its path around another celestial body.
    \item Obliquity of the ecliptic: angle between the ecliptic and the celestial equator.
    \item Vernal Equinox: equinox of the Earth when the subsolar point crosses the equator northward. 
\end{itemize}

\subsection*{c.}
Due to the continuous precession of the Earth's rotation axis, the Earth's vernal equinox moves on the celestial plane. It is caused by the Sun and Moon pulling on the equatorial bulge of the Earth.
The amplitude of the Sun-Moon precession is about 23 degrees, the amplitude of the Sun-Moon nutation is about 9''. The Sun-Moon precession and the planetary precession do not influence the obliquity, it is the nutation which does.

\subsection*{d.}
The following terms need to be explained:
\begin{itemize}
    \item Sidereal Time: time system based on the Earth's rotation with respect to the stars
    \item Solar Time: time system based on the Sun's position in the sky
    \item Mean Solar Time: hour angle of the sun plus 12 hours
    \item Atomic Time: time system based on hyperfine transition frequency of atoms such as Caesium-123
    \item Universal Time: time system based on the Earth's rotation with respect to distant stars, but with additional corrections to make it closer to solar time
    \item Universal Time Coordinated: time system based on atomic time with leap seconds added to make it within 1 second of UT1.
\end{itemize}